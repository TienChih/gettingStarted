\documentclass{ximera}
\usepackage[colorlinks=true,urlcolor=blue]{hyperref}
\title{Troubleshooting}
\begin{document}
\begin{abstract}
This activity contains some possible solutions
to issues that might arise
\end{abstract}
\maketitle
\begin{enumerate}
\item Be careful when pasting the key into
the webhook secret field. Be sure that no extra
spaces were copied. Push the \verb!Update webhook! button on
\href{http://github.org}{\tt github.org}
after correcting the secret.

\item To see updates on the course webpage
after pushing changes through \verb!git!,
push the browser's refresh button
and then click the \verb!Update activity! button 
on the course page. Deleting your browser's cache
might also resolve certain issues.

\item Use \verb!\includegraphics{file.png}! rather than
\verb!\includegraphics{file}!. \LaTeX\ will gracefully handle either
case whereas Ximera currently requires explicit file extensions.

\item The optional scaling argument of \verb!\includegraphics!
currently does not work on the Ximera side.

\item The preamble of every activity file needs to 
include a \verb!\title! command. 

\item The abstract 
and the \verb!\maketitle! command
of every activity file
need to occur after \verb!\begin{document}!.

\item A good way to check for \TeX\ typing mistakes
is to run \LaTeX\ on all the \TeX\ files. Indeed, this
is good practice irrespective of mistakes
and additionally produces handouts that
can be given to students.

\item If your \TeX\ files compile correctly,
you should confirm that they have been successfully uploaded to
\href{http://github.org}{\tt github.org} and that 
\href{http://github.org}{\tt github.org} has successfully
pushed those files in turn to
\href{http://ximera.osu.edu}{\tt ximera.osu.edu}.
To do this you can click on the most recent delivery
on the \verb!Webhooks & Services! page in your repository on
\href{http://github.org}{\tt github.org}.
In case of any delivery errors, they will be reported here.
You can also push the \verb!Redeliver! button
in case of any errors.

\end{enumerate}
\end{document}
