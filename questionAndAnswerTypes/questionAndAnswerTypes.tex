\documentclass{ximera}
\title{Question and answer types}
\begin{document}

\begin{abstract}
  This section explores the different types of exercises Ximera supports. 
\end{abstract}

\maketitle

Ximera supports many exercise types.


\section{Basic Function Answer Type}	
\begin{verbatim}
         $3\times 2 = \answer{6}$
\end{verbatim}

Will produce the question:

\begin{question}
  $3\times 2 = \answer{6}$
\end{question}

In addition to numerical answers, we also support elementary functions:

\begin{verbatim}
    \begin{question}
         $ \frac{\partial}{\partial x} x^2\sin(y) =  \answer{2x\sin(y)}$
    \end{question}
\end{verbatim}

Produces:

\begin{question}
  $ \frac{\partial}{\partial x} x^2\sin(y) = \answer{2x\sin(y)}$
\end{question}

\begin{remark}
Under the hood, Ximera is parsing the user input, producing a
function, and checking the user input function against ``answer'' at
$100$ different complex numbers, and seeing if the results are
``reasonably'' close to each other.  We compare the complex extensions
of these functions to circumvent domain issues.
\end{remark}

\section{Multiple Choice Answer Type}

\begin{verbatim}
\begin{question}
	Which of the following functions has a graph which is a parabola?
	\begin{multipleChoice}
		\choice[correct]{$y=x^2+3x-3$}
		\choice{$y = \frac{1}{x+2}$}
		\choice{$y=3x+1$}
	\end{multipleChoice}
\end{question}
\end{verbatim}

Produces:

\begin{question}
  Which of the following functions has a graph which is a parabola?
  \begin{multipleChoice}
    \choice[correct]{$y=x^2+3x-3$}
    \choice{$y = \frac{1}{x+2}$}
    \choice{$y=3x+1$}
  \end{multipleChoice}
\end{question}

\begin{remark}
  Multiple choice questions are automatically shuffled.
\end{remark}

\section{Matrix Answer Type}

We also support matrices of expressions.

\begin{verbatim}
\begin{question}
Enter the matrix  \(\begin{bmatrix} x & y \\ xy & z+1 \end{bmatrix}\)
    \begin{matrixAnswer}
	    correctMatrix = [['x','y'],['xy','z+1']]
    \end{matrixAnswer}
\end{question}
\end{verbatim}

\begin{question}
  Enter the matrix  \(\begin{bmatrix} x & y \\ xy & z+1 \end{bmatrix}\)
  \begin{matrixAnswer}
    correctMatrix = [['x','y'],['xy','z+1']]
  \end{matrixAnswer}
\end{question}

\begin{remark}
  The plus and minus buttons add and subtract columns or rows.  
\end{remark}

\section{free-response}

The free response environment gives students access to a \LaTeX\ editor. 

\begin{verbatim}
\begin{question}
	Question goes here!
	\begin{freeResponse}
	This is the model solution %You don't actually need anything in between the begin and end line.
	\end{freeResponse} 
\end{question}
\end{verbatim}

\begin{question}
	Question goes here!
	\begin{freeResponse}
	This is the model solution %You don't actually need anything in between the begin and end line.
	\end{freeResponse} 
\end{question}

\begin{remark}
Clicking on \verb!View model solution! shows the user
whatever you typed in the  \verb!freeResponse! environment.
\end{remark}

\end{document}
