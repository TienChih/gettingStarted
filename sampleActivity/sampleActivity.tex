\documentclass{ximera}
\title{Sample Activity}
\begin{document}

	\begin{abstract}
		This section teaches how to use some of the features of Ximera
	\end{abstract}
	
	Ximera supports several exercise types.
	
	
\section{Basic Function Answer Type}	
\begin{verbatim}
    \begin{question}
        \begin{solution}
            $3\times 2 = $ \answer{6}
        \end{solution}
    \end{question}
\end{verbatim}

Will produce the question:

\begin{question}
        \begin{solution}
            $3\times 2 = $ \answer{6}
        \end{solution}
\end{question}

In addition to numerical answers, we also support elementary functions:

\begin{verbatim}
    \begin{question}
        \begin{solution}
            $ \frac{\partial}{\partial x} x^2\sin(y) = $ \answer{2xsin(y)}
        \end{solution}
    \end{question}
\end{verbatim}

Produces:

  \begin{question}
        \begin{solution}
            $ \frac{\partial}{\partial x} x^2\sin(y) = $ \answer{2xsin(y)}
        \end{solution}
    \end{question}
    
Under the hood, Ximera is parsing the user input, producing a function, and checking the user input function against "answer" at 
$100$ different complex numbers, and seeing if the results are "reasonably" close to each other.  
We compare the complex extensions of these functions to circumvent domain issues.

\section{Multiple Choice Answer Type}

\begin{verbatim}

\begin{question}
	Which of the following functions has a graph which is a parabola?
	\begin{multiple-choice}
		\choice[correct]{$y=x^2+3x-3$}
		\choice{$y = \frac{1}{x+2}$}
		\choice{$y=3x+1$}
	\end{multiple-choice}
\end{question}
\end{verbatim}

Produces:

\begin{question}
	Which of the following functions has a graph which is a parabola?
	\begin{multiple-choice}
		\choice[correct]{$y=x^2+3x-3$}
		\choice{$y = \frac{1}{x+2}$}
		\choice{$y=3x+1$}
	\end{multiple-choice}
\end{question}

Currently, multiple choice questions are automatically shuffled.

\section{Matrix Answer Type}

We also support matrices of expressions.

\begin{verbatim}
\begin{question}
Enter the matrix  \(\begin{bmatrix} x & y \\ xy & z+1 \end{bmatrix}\)
\begin{solution}
    \begin{matrix-answer}
	    correctMatrix = [['x','y'],['xy','z+1']]
    \end{matrix-answer}
\end{solution}
\end{question}
\end{verbatim}

\begin{question}
Enter the matrix  \(\begin{bmatrix} x & y \\ xy & z+1 \end{bmatrix}\)
\begin{solution}
    \begin{matrix-answer}
	    correctMatrix = [['x','y'],['xy','z+1']]
    \end{matrix-answer}
\end{solution}
\end{question}

The plus and minus muttons add and subtract columns or rows.  

\section{free-response}

The free response environment gives students access to a latex editor. 

\begin{free-response}
	This is the model solution.
\end{free-response}

Clicking on ''view model solution" lets the students see whatever you typed in the ''free-response" environment.






	
	
\end{document}
