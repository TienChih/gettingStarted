\documentclass{ximera}
\begin{document}
	\title{Sample Activity}
	\begin{abstract}
		This section teaches how to use some of the features of Ximera
	\end{abstract}
	
	Ximera supports several exercise types.  For example the code
	
\begin{verbatim}
    \begin{question}
        \begin{solution}
            $3\times 2 = $ \answer{6}
        \end{solution}
    \end{question}
\end{verbatim}

Will produce the question:

\begin{question}
        \begin{solution}
            $3\times 2 = $ \answer{6}
        \end{solution}
\end{question}

In addition to numerical answers, we also support elementary functions:

\begin{verbatim}
    \begin{question}
        \begin{solution}
            $ \frac{\partial}{\partial x} x^2\sin(y) = $ \answer{2xsin(y)}
        \end{solution}
    \end{question}
\end{verbatim}

Produces:

  \begin{question}
        \begin{solution}
            $ \frac{\partial}{\partial x} x^2\sin(y) = $ \answer{2xsin(y)}
        \end{solution}
    \end{question}
    
Under the hood, Ximera is parsing the user input, producing a function, and checking the user input function against "answer" at 
$100$ different complex numbers, and seeing if the results are "reasonably" close to each other.  
We compare the complex extensions of these functions to circumvent domain issues.




	
	
\end{document}