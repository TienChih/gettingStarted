\documentclass{ximera}
\begin{document}
\title{Setting up the repository}
\begin{enumerate}
\item Create a directory and change to that directory.
In this example, we will create a director called gettingStarted.
\begin{center}
\begin{verbatim}
mkdir gettingStarted && cd gettingStarted
\end{verbatim}
\end{center}
\item A Ximera directory should contain
a text file called \verb!course.xim! containing
at least the name of the course, a description of the course,
and the names of all the activity \LaTeX\ files in the order
they should be presented. In the example below
there is one activity file \verb!repoSetup.tex!
written without the extension \verb!.tex!,
which lives in a directory \verb!repoSetup!.
\begin{verbatim}
---
name: Getting Started with Ximera
description: This is a Ximera activity explaining how to get started
with Ximera for instructors.
---

repoSetup/repoSetup
\end{verbatim}
In general, it is a sensible policy to have each
activity in its own directory, of the same name.

\end{enumerate}
\end{document}
